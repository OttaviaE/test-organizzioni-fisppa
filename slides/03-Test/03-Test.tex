\documentclass[compress]{beamer}
\usetheme{Montpellier}
\useinnertheme{rounded}
\useoutertheme{miniframes}
\usecolortheme{dove}
\usepackage{graphicx}
\usepackage{setspace}
\usepackage{tabularx}
\usepackage[italian]{babel}
\usepackage{tikzsymbols}
\usepackage{tikz}
\usepackage{spot}
\usepackage{tabularx}
\usepackage[absolute,overlay]{textpos}
\usepackage{booktabs}
\usepackage{tikz}
\usepackage{xcolor}
\usepackage[utf8]{inputenc}
\setbeamertemplate{caption}[numbered]
\usetikzlibrary{mindmap,calc,patterns,decorations.pathmorphing,decorations.markings, arrows, shapes.arrows, shapes, backgrounds,positioning,shadows.blur}
\usepackage[absolute,overlay]{textpos}
\usetikzlibrary{shapes.geometric, arrows}
\tikzstyle{latent} = [ellipse, minimum width=2.5cm, minimum height=2cm,text centered, draw=black]
\tikzstyle{item} = [rectangle, rounded corners, minimum width=0.5cm, minimum height=2cm, text width =1.7cm, text centered, draw=black]
\tikzstyle{arrow} = [thick,->,>=stealth]
\def\tikzoverlay{%
	\tikz[remember picture, overlay]\node[every overlay node]
}%

\setbeamersize{text margin left=10pt, text margin right=10pt}
\newcommand\Factor{1.2}
\definecolor{unipd}{RGB}{155, 0, 20}
\definecolor{grigioPantano}{RGB}{72,79,89}
\definecolor{blu}{rgb}{0.0, 0.33, 0.71}
\setbeamerfont{block title}{family=\bfseries}
\setbeamercolor{block title}{use=structure,fg=unipd,bg=white}
  \AtBeginSection[]
  {
    \begin{frame}[plain]
      \tableofcontents[currentsection, hideallsubsections]
    \end{frame}
  }

  \AtBeginSubsection[]
  {
    \begin{frame}[plain]
      \tableofcontents[currentsection,currentsubsection,hideothersubsections]
    \end{frame}
  }
  
\setbeamertemplate{footnote}{%
	\parindent 1em\noindent%
	\raggedright
	\insertfootnotetext\par%
}

  \tikzset{
	invisible/.style={opacity=0},
	visible on/.style={alt={#1{}{invisible}}},
	alt/.code args={<#1>#2#3}{%
		\alt<#1>{\pgfkeysalso{#2}}{\pgfkeysalso{#3}} % \pgfkeysalso doesn't change the path
	},
	every overlay node/.style={
		%draw=black,fill=white,rounded corners,
		anchor=north west, inner sep=0pt,
	},
}

\def\tikzoverlay{%
	\tikz[remember picture, overlay]\node[every overlay node]
}%


\title[Misura in $\psi$]{Modulo Didattico 3: Test, scale di risposta, validità e attendibilità}
\subtitle{Test per le organizzazioni}
\date{A.A.: 2025/2026}
\author[Ottavia Epifania]{\texorpdfstring{Dr. Ottavia M. Epifania\newline\url{ottavia.epifania@unipd.it}}{Author}}
\institute[]{Università di Padova}
\author{\textbf{Ottavia M. Epifania}\\\texttt{ottavia.epifania@unipd.it}\\
Margherita Calderan\\\texttt{margherita.calderan@unipd.it}}

\begin{document}

\begin{frame}[plain]
\maketitle
\end{frame}

\section[Scale di Risposta]{Scale di Risposta}

\begin{frame}{Item Vero/Falso}
	Sono item formulati in modo che l'unica possibile risposta possa essere Sì/No o Vero/Falso 
	
	\begin{quote}{Esempio}
		
		Mi commuovo quando vedo un film drammatico.
		\begin{itemize}
			\item Sì
			\item No
		\end{itemize}
	\end{quote}
	
	\begin{block}{Vantaggi}
		
		Sono estremamente facili da comprendere per chiunque 
	\end{block}
	
	\begin{block}{Svantaggi}
		
		La scala di risposta potrebbe essere troppa riduttiva e particolarmente complessa per certe tipologie di persone
	\end{block}
\end{frame}

\begin{frame}{Item a scelta multipla forzata}
	
	\begin{quote}
		
		Preferisco un lavoro nel quale: 
		
		\begin{enumerate}
			\item Posso crescere come persona
			\item Posso guadagnare bene
			\item Posso imparare cose nuove
		\end{enumerate}
	\end{quote}
	
		
	\vspace{2.5mm}
	Le opzioni di risposta sono realmente ordinabili? Sono anche solo comparabili?
	
	\pause
	\begin{block}{Narcissistic Personality Inventory}
		
		\pause
		\begin{enumerate}
			\item non sono né meglio né peggio di altre persone
			\item penso di essere una persona speciale 
		\end{enumerate}
		
		Una persona con un alto livello di narcisismo ha più probabilità di scegliere la seconda opzione
	\end{block}
\end{frame}

\begin{frame}{Rating scales}
	
	L'evoluzione delle scale con item Sì/No o Vero/Falso $\rightarrow$ aggiunta di livelli intermedi
	
	\emph{Summating rating scales:} Scale composte da diverse domande (\textbf{item}) il cui punteggio totale è ottenuto attraverso la somma dei punteggi alle valutazioni fornite ad ogni item
	
	\pause
	\begin{block}{Assunzione}
		
		Il costrutto latente si muove lungo un continuum sottostante alle opzioni di risposte, ovvero è possibile dare una quantità alla variabile psicologica misurata a seconda dell'opzione di risposta scelta. 
		\end{block}
		

\end{frame}

\begin{frame}

		
		Esprimere il proprio livello di accordo con l'affermazione ``Sono una persona organizzata'' secondo le seguenti opzioni di risposta: 
		
		\begin{center}
			Per niente d'accordo:	$\square$ \hspace{2mm} $\square$ \hspace{2mm} $\square$ \hspace{2mm} $\square$ \hspace{2mm} $\square$: Completamente d'accordo
		\end{center}

L'assunzione è che il livello di accordo con l'affermazione vari lungo un continuum e che questo continuum sia delimitato dagli ancoraggi ``Per niente d'accordo'' e ``Completamente d'accordo'': 

\pause
\begin{figure}
	\centering
	\includegraphics[width=0.9\linewidth]{img/accordo}
\end{figure}

\end{frame}

\begin{frame}{``Sono una persona organizzata''}{Indicare il grado di accordo con l'affermazione}
	
	\begin{block}{Barrare il valore numerico corrispondente}
		
		\begin{tabular}{p{2cm} p{2cm}  p{2cm}  p{2cm}  p{2cm} }
			Per niente & & & & Completamente \\
			1 & 2 & 3 & 4& 5 \\
		\end{tabular}
	\end{block}
	
		\begin{block}{Barrare la casella}
			
		\begin{tabular}{p{2cm} p{2cm}  p{2cm}  p{2cm}  p{2cm} }
			Per niente & & & & Completamente \\
			$\square$ & $\square$ & $\square$ & $\square$& $\square$ \\
		\end{tabular}
	\end{block}
	
\begin{block}{Segnare il numero}
	
	``Sono una persona organizzata'': \hspace{2cm} \large $\square$
\end{block}

	
\end{frame}

\begin{frame}{Visual Analougue Scale}

Molto utilizzate anche in psicologia dello sviluppo 

	\begin{figure}
		\centering
		\includegraphics[width=0.7\linewidth]{img/vas}
	\end{figure}

\end{frame}

\begin{frame}{Domini}
	
	\begin{block}{Accordo}
		
		Indicare il grado di accordo con un'affermazione
		
		Ad esempio:\\
	\emph{Sono una persona organizzata} 
	\begin{center}
		Per niente d'accordo:	$\square$ \hspace{1.5mm} $\square$ \hspace{1.5mm} $\square$ \hspace{1.5mm} $\square$ \hspace{1.5mm} $\square$: Completamente d'accordo
	\end{center}
	\end{block}
	
		\begin{block}{Intensità}
		
	Dare una valutazione vera e propria rispetto a un argomento in termini di buono/cattivo, insufficiente/sufficiente, importante/non importante, eccetera. 
	
	Ad esempio: \\
	\emph{Indicare il proprio livello di competenze informatiche}
	\begin{center}
		5 $=$ alto; \hspace{1.5mm} 4 $=$ medio-alto; \hspace{1.5mm} 3 $=$ medio; \hspace{1.5mm} 2 $=$ medio-basso; \hspace{1.5mm} 1 $=$ basso
	\end{center}
	\end{block}
	

\end{frame}

\begin{frame}
	\small
		\begin{block}{Frequenza}
		
		Indicare la frequenza con cui vengono messi in atto i comportamenti o vengono sperimentati i vissuti descritti dall'item
		
		Ad esempio: \\
		\emph{Nell'ultimo anno, mi sono sentito ansioso:}
		\begin{center}
	Mai	$\square$ \hspace{1.5mm} Raramente $\square$ \hspace{1.5mm} Qualche volta $\square$ \hspace{1.5mm} Spesso $\square$ \hspace{1.5mm} Sempre $\square$
		\end{center}
	\end{block}
	
	\pause
	\begin{center}
		Attenzione!
	\end{center}
	
	Cosa vuole dire ``Spesso''? E ``Qualche volta''? Vogliono dire la stessa cosa per tutt*?
	
	\begin{columns}[T]
		\begin{column}{.50\linewidth}
			\begin{center}
				Bassa frequenza
			\end{center}
			\begin{enumerate}
				\item Mai 
				\item Circa una volta l'anno
				\item Circa due volte l'anno
				\item Due volte al mese 
				\item Più di due volte al mese
			\end{enumerate}
		\end{column}
		\begin{column}{.50\linewidth}
					\begin{center}
			Alta frequenza
		\end{center}
		\begin{enumerate}
			\item Due volte al mese o meno 
			\item Una volta a settimana
			\item Due volte a settimana
			\item Tutti i giorni 
			\item Diverse volte al giorno
		\end{enumerate}
		\end{column}
	\end{columns} 
\end{frame}

\begin{frame}{Quanti punti?}
	Come regola generale, più è alto il numero di punti della scala, maggiore è poi l'attendibilità della misura
	
	Solitamente si considerano scale con un numero di punti tra 4 e 7 $\rightarrow$ Più di 7 punti sono ``Inutili''
	
	Numero dispari o numero pari di punti? 
	
	Un numero dispari di punti permette l'alternativa neutra: 
	
	\begin{itemize}
		\item Ci sono dei casi (e.g., valutazione di soddisfazione per un servizio) dove ha senso avere un'alternativa neutra 
		\item In altri casi (e.g., valutazione degli atteggiamenti verso un gruppo sociale) l'alternativa neutra funge da ``rifugio'' per non sbilanciarsi
	\end{itemize}
\end{frame}


\begin{frame}{Criticità}
	\begin{block}{Acquiescienza}
		
		Essere sistematicamente d'accordo con le affermazioni proposte dagli item
	\end{block}
	
\begin{block}{Estremismo}
	
	Scegliere sempre le risposte estreme
\end{block}

\begin{block}{Evasività}
	
	Scegliere sempre i punti centrali della scala o le risposte ``Non so''
\end{block}

\begin{block}{Desiderabilità sociale}
	
Rispondere in modo da mostrarsi delle belle persone sempre e comunque	
\end{block}

\end{frame}

\section[Validità]{Validità e attendibilità delle misure}

\begin{frame}{Attendibilità}
	
	Grado in cui una procedura di misurazione produce lo stesso risultato in prove ripetute
	
	\begin{equation*}
		X = V + E
	\end{equation*}
	dove: \\
	$X$: misura rilevata\\
	$V$: parte vera\\
	$E$: errore: fluttuazioni casuali oppure costante e sistematico
	
	\pause
	L'attendibilità di una misura è la proporzione di $X$ che non riflette l'errore di misurazione: 
	\begin{equation*}
		\rho = \dfrac{V}{V + E}
	\end{equation*}
	
\end{frame}

\begin{frame}{Una nota sull'errore}
	\begin{block}{Errore casuale}
		
		Tutti quei fattori casuali che confondono la misurazione di qualunque fenomeno. \\
		Caratteristiche: 
		\begin{itemize}
			\item Ha media zero
		\item 	La correlazione fra punteggio vero ed errore è zero
			\item La correlazione fra errore e punteggio vero alla misurazione successiva è zero
			\item La correlazione fra errori di misurazioni diverse è zero
			
		\end{itemize}
		
	\end{block}
	
		\begin{block}{Errore sistematico (bias)}
		
		Effetto distorcente non casuale, sistematico
	\end{block}
\end{frame}

\begin{frame}

\begin{block}{Validità di costrutto}
	
	Quello che si sta misurando... è proprio quello che si voleva misurare e non altro!
	
	(Alcuni) Elementi utili per la \underline{validità di costrutto}:
	\begin{itemize}
		\item Chiara definizione del costrutto teorico
		\item Misurare il costrutto con metodi differenti
	\end{itemize} 
\end{block}	

\pause
\begin{block}{Validità statistica}
	
	Quello che stiamo iferendo a partire dalle analisi ha senso perché le analisi hanno senso! 
	
	(Alcuni) Elementi utili per la \underline{validità statistica}:
	\begin{itemize}
		\item Appropriatezza dei metodi statistici utilizzati sulla base delle variabili misurate
		\item Adeguatezza dell'ampiezza campionaria
	\end{itemize} 
	
\end{block}
\end{frame}

\begin{frame}
	\begin{block}{Validità esterna}
		
		Il grado in cui i risultati sono \emph{rappresentativi} della popolazione di interesse e il grado in cui sono \emph{replicabili}
		
			(Alcuni) Elementi utili per la \underline{validità esterna}:
		\begin{itemize}
			\item Il campione è rappresentativo della popolazione target
			\item La validità ecologica
		\end{itemize} 
	\end{block}
	
	\pause
		\begin{block}{Validità ecologica}
		
		Rappresenta la generalizzabilità dei risultati di una ricerca a quella che è la vita quotidiana 
		
		I dati raccolti devono essere rappresentativi del comportamento dell’individuo nella sua realtà abituale
		
	\end{block}
\end{frame}

\end{document}