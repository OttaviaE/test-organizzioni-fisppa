\documentclass[compress]{beamer}
\usetheme{Montpellier}
\useinnertheme{rounded}
\useoutertheme{miniframes}
\usecolortheme{dove}
\usepackage{graphicx}
\usepackage{setspace}
\usepackage{tabularx}
\usepackage[italian]{babel}
\usepackage{tikzsymbols}
\usepackage{tikz}
\usepackage{spot}
\usepackage{tabularx}
\usepackage[absolute,overlay]{textpos}
\usepackage{booktabs}
\usepackage{tikz}
\usepackage{xcolor}
\usepackage[utf8]{inputenc}
\setbeamertemplate{caption}[numbered]
\usetikzlibrary{mindmap,calc,patterns,decorations.pathmorphing,decorations.markings, arrows, shapes.arrows, shapes, backgrounds,positioning,shadows.blur}
\usepackage[absolute,overlay]{textpos}
\usetikzlibrary{shapes.geometric, arrows}
\tikzstyle{latent} = [ellipse, minimum width=2.5cm, minimum height=2cm,text centered, draw=black]
\tikzstyle{item} = [rectangle, rounded corners, minimum width=0.5cm, minimum height=2cm, text width =1.7cm, text centered, draw=black]
\tikzstyle{arrow} = [thick,->,>=stealth]
\def\tikzoverlay{%
	\tikz[remember picture, overlay]\node[every overlay node]
}%

\setbeamersize{text margin left=10pt, text margin right=10pt}
\newcommand\Factor{1.2}

  \AtBeginSection[]
  {
    \begin{frame}[plain]
      \tableofcontents[currentsection, hideallsubsections]
    \end{frame}
  }

  \AtBeginSubsection[]
  {
    \begin{frame}[plain]
      \tableofcontents[currentsection,currentsubsection,hideothersubsections]
    \end{frame}
  }
  
\setbeamertemplate{footnote}{%
	\parindent 1em\noindent%
	\raggedright
	\insertfootnotetext\par%
}

  \tikzset{
	invisible/.style={opacity=0},
	visible on/.style={alt={#1{}{invisible}}},
	alt/.code args={<#1>#2#3}{%
		\alt<#1>{\pgfkeysalso{#2}}{\pgfkeysalso{#3}} % \pgfkeysalso doesn't change the path
	},
	every overlay node/.style={
		%draw=black,fill=white,rounded corners,
		anchor=north west, inner sep=0pt,
	},
}

\def\tikzoverlay{%
	\tikz[remember picture, overlay]\node[every overlay node]
}%


\title[Misura in $\psi$]{Modulo Didattico 1: Il concetto di Misura (in psicologia)}
\subtitle{Test per le organizzazioni}
\date{A.A.: 2025/2026}
\author[Ottavia Epifania]{\texorpdfstring{Dr. Ottavia M. Epifania\newline\url{ottavia.epifania@unipd.it}}{Author}}
\institute[]{Università di Padova}
\author{\textbf{Ottavia M. Epifania}\\\texttt{ottavia.epifania@unipd.it}\\
Margherita Calderan\\\texttt{margherita.calderan@unipd.it}}

\begin{document}

\begin{frame}[plain]
\maketitle
\end{frame}

\section{Il corso}

%------------------------------------------------
\begin{frame}{Orari}

\textbf{Lezioni:}
\begin{itemize}
    \item Lunedì: 12:30--15:30
\end{itemize}

\vspace{0.3cm}

\textit{Lunedì 6 aprile non ci sarà lezione}

\vspace{0.5cm}

\textbf{Ricevimento:} su appuntamento, prevalentemente online

\vspace{0.5cm}

\begin{center}
ottavia.epifania@unipd.it \\
margherita.calderan@unipd.it
\end{center}

\end{frame}

%------------------------------------------------
\begin{frame}{Argomenti}

Il corso prevede parti di natura teorica relative agli aspetti della misura in psicologia (nel caso particolare dei test nel contesto organizzativo) e esercitazioni pratiche, dove vengono illustrate le procedure più comuni per l'analisi della dimensionalità dei test e delle loro caratteristiche.

\vspace{0.4cm}

\textbf{Alcune delle tematiche:}
\begin{itemize}
    \item Misurazione, scale di misura e statistiche adeguate
    \item Rappresentazioni grafiche del dato
    \item Diversi tipi di test per diversi obiettivi misurativi
    \item Teoria classica dei test: approccio esplorativo e confermativo
    \item Applicazioni in R
\end{itemize}

\end{frame}

%------------------------------------------------
\begin{frame}{Esame}

L'esame avrà durata di 90 minuti e sarà composto da 20 domande:
\begin{itemize}
    \item 12 a risposta chiusa
    \item 8 di interpretazione di output ottenuti con R
\end{itemize}

\vspace{0.4cm}

Nelle domande a risposta aperta verranno valutati i seguenti criteri:

\begin{itemize}
    \item Chiarezza espositiva
    \item Conoscenza dell'argomento
    \item Capacità di veicolare concetti complessi
\end{itemize}

\end{frame}

%------------------------------------------------
\section{La misura in psicologia}

%------------------------------------------------
\begin{frame}{}

\begin{block}{Compito!}
Siete responsabili HR in un'azienda e dovete scegliere un* nuov* project manager tra una rosa di candidati. 

Quali caratteristiche di personalità ritenete fondamentali in un PM?
\end{block}

\pause

\textbf{Alcune possibilità:}
\begin{itemize}
    \item Leadership
    \item Stabilità emotiva
    \item Coscienziosità
    \item Teamwork
    \item Resilienza
    \item Assertività
\end{itemize}

\end{frame}


\begin{frame}
\Large
\centering
Cos'è la leadership? 

Come si misura la leadership? 

\end{frame}


\begin{frame}

\vspace{-5mm}
\begin{block}{Misura}

Processo di assegnazione di numeri o simboli a proprietà di oggetti, eventi o fenomeni, seguendo regole definite e condivise per rappresentarne quantitativamente le proprietà e consentirne la descrizione, il confronto e l’analisi.

\end{block}

\vspace{-2mm}

\begin{columns}[T]
\begin{column}{.50\linewidth}
\begin{center}
Quantità
\end{center}
\small 

\textbf{Misurazione}
			
La \textbf{misura} esprime una manifestazione particolare della quantità
			
Le misure sono esprimibili con numeri che rappresentano una quantità
			
\begin{exampleblock}{Esempio:}

\small				
La larghezza della cattedra è una misura della quantità ``lunghezza''
\end{exampleblock}
\end{column}

\begin{column}{.50\linewidth}
\begin{center}
Qualità
\end{center}
\small
			
\textbf{Rilevazione}
			
Caratteristiche che variano da un'unità di analisi all'altra ma non sono quantità 
			
Le rilevazioni si esprimono con simboli (anche numerici) ma \textbf{i numeri non esprimono misure di quantità}
			
\begin{exampleblock}{Esempio:}

\small				
Luogo di nascita degli studenti di questa classe
\end{exampleblock}
\end{column}
\end{columns}
\end{frame}

\begin{frame}
	
	\begin{tikzpicture}
		\draw[-,thick] (0,-6) -- (8,-6) node[left, xshift =3cm] {bastoncino $a$};
		\draw[-,thick] (0,-7) -- (4,-7) node[left, xshift =3cm] {bastoncino $b$};
	\end{tikzpicture}
	
	Il bastoncino $a$ è lungo il doppio del bastoncino $b$: 
	
	\begin{block}%
		
		$b = \dfrac{1}{2}a$ \hspace{1.5mm} oppure $a = 2b$
		
		\vspace{1.5mm}
		Se aggiungiamo una costante $k$ ad $a$ e $b$, il rapporto tra $a$ e $b$ non cambia
	\end{block}
	
	Il sistema relazionale numerico è in relazione diretta con il sistema relazione empirico
	
	\pause
	\vspace{2mm}
	Le variabili psicologiche non fanno parte del sistema relazionale empirico! Come si possono quantificare? Come si possono misurare?
\end{frame}


\begin{frame}
	\begin{center}
		Cos'è la {\Large{leadership}}?
		\pause
		\\\vspace{1.5mm} Si può vedere la leadership?
	\end{center}
	
	\pause
	La leadership di per sé non si può vedere, non si può toccare, \textbf{non si può osservare direttamente} $\rightarrow$ non è nel sistema relazionale empirico
	
	\vspace{1.5mm}
	
	Quello che si può vedere direttamente sono i \textbf{comportamenti} che possono essere ricondotti alla leadership $\rightarrow$ sono nel sistema relazionale empirico e si possonono misurare 
	
	Si ``traduce'' la definizione teorica di una variabile/caratteristica psicologica in qualcosa di osservabile $\rightarrow$ i comportamenti (che possono anche essere le risposte a un questionario!)
\end{frame}

\begin{frame}

	\begin{figure}
		\centering
		\includegraphics[width=.9\linewidth]{img/misura.png}
	\end{figure}
	
\vskip0pt plus 1filll

\color{black}\rule{0.30\linewidth}{0.5pt}\\
\color{black}
\scriptsize{Immagine adattata da Chiorri, C. (2023). \emph{Teoria e Tecnica Psicometrica, 2 ed.}, McGraw-Hill}

\end{frame}

\subsection*{Variabli osservate vs. Variabili latenti}


\begin{frame}

\begin{itemize}
\item Variabili che \textit{non possono essere osservate direttamente} $\rightarrow$ \textbf{Variabili latenti} (es. Intelligenza)

\medskip

\item Inferite da indicatori osservabili direttamente $\rightarrow$ \textbf{Variabili osservate} (es. risposte alle Matrici di Raven)

\end{itemize}
\pause

L'\textit{operazionalizzazione} della variabile latente è cruciale


\end{frame}

\begin{frame}

Le variabili latenti devono essere collegate alle variabili osservate $\rightarrow$ modelli matematici e statistici

\medskip

\textbf{Assunzioni:}

\begin{itemize}
\item Le variabili latenti sono la causa sottostante delle variabili osservate

\medskip

\item \textit{Indipendenza locale}: la correlazione tra variabili osservate scompare dopo aver controllato l'influenza della variabile latente
\end{itemize}

\centering
\includegraphics[width=.70\linewidth]{img/latent.png}

\end{frame}


\begin{frame}

\centering
\includegraphics[width=.90\linewidth]{img/mavl1.png}

\end{frame}

\subsection*{Operazionalizzazione}

\begin{frame}
	
	Si tratta della ``traduzione'' in comportamenti osservabili ed oggettivi di variabili latenti psicologiche \emph{non direttamente osservabili}
	
	La definizione teorica del costrutto diventa di vitale importanza per la definizione dei comportamenti osservabili ad esso legati... La misurazione del costrutto dipende proprio da questi!
	
	\begin{block}{Dominio di contenuto}
		
		Universo dei possibili comportamenti che, coerentemente con la definizione, possono rappresentare le operazionalizzazioni del costrutto 
		
		Quando è molto ampio $\rightarrow$ \emph{facets}
		
	\end{block}
\end{frame}

\subsection*{Estroversione}
\begin{frame}

\textbf{Definizione}

L'estroversione è un tratto di personalità caratterizzato da socievolezza, assertività, livello elevato di attività, ricerca di stimoli ed esperienza di emozioni positive 
{\footnotesize (Costa \& McCrae, 1992)}.

\begin{spacing}\Factor
\begin{table}
\begin{tabular}{l l}
\hline
Facets	&	Aggettivi prototipici	\\\hline
Socievolezza	&	Socievole	\\
Assertività	&	Deciso	\\
Attività	&	Energico	\\
Ricerca di stimoli 	&	Avventuroso	\\
Emozioni positive	&	Entusiastico	\\
Espansività	&	Espansivo	\\\hline
\end{tabular}
\end{table}
\end{spacing}

\end{frame}

\begin{frame}
	\begin{table}
			\begin{tabular}{l l l}
			\multicolumn{1}{c}{\textbf{Socievolezza}}	&	\multicolumn{1}{c}{\textbf{Espansività}}	&	\multicolumn{1}{c}{\textbf{Ricerca di stimoli}}	\\
			Socievole	&	Espansivo	&	Avventuroso	\\
			Spontaneo	&	Amichevole	&	Intrapredente	\\
			Simpatico	&	Schietto	&	Audace	\\
			Festaiolo	&	Sincero	&	Sfacciato	\\
			\emph{Chiuso}	&	Aperto	&	Coraggioso	\\
			\emph{Riservato}	&	Comunicativo	&	\emph{Fifone}	\\
			\multicolumn{1}{c}{\textbf{Assertività}}	&	\multicolumn{1}{c}{\textbf{Emozioni positive}}	&	\multicolumn{1}{c}{\textbf{Attività}}	\\
			Deciso	&	Entusiatsico	&	Energico	\\
			Determinato	&	Brioso	&	Dinamico	\\
			Sicuro di sé	&	Gioioso	&	Attivo	\\
			Fiducioso	&	Solare	&	Vigoroso	\\
			Risoluto	&	Spensierato	&	Rapido	\\
			\emph{Timido}	&	\emph{Preoccupato}	&	\emph{Letargico}	\\
			
			
		\end{tabular}
	\end{table}

\end{frame}

\subsection*{Leadership}

\begin{frame}

\textbf{Definizione}

La leadership è il processo attraverso cui un individuo influenza un gruppo di persone al fine di raggiungere un obiettivo comune 
{\footnotesize (Northouse, 2021)}.

\begin{spacing}\Factor
\begin{table}
\begin{tabular}{l l}
\hline
Dimensioni	&	Aggettivi prototipici	\\\hline
Influenza sociale	&	Persuasivo	\\
Visione	&	Ispiratore	\\
Decisione	&	Determinato	\\
Gestione del gruppo 	&	Coordinatore	\\
Motivazione	&	Coinvolgente	\\
Responsabilità	&	Affidabile	\\\hline
\end{tabular}
\end{table}
\end{spacing}

\end{frame}

\subsection*{Triade Oscura}

\begin{frame}

Alcuni costrutti di personalità sono concettualizzati come dimensioni relativamente unitarie, senza una struttura gerarchica stabile composta da facets.

\medskip

Un esempio è la \textbf{Triade Oscura della personalità}, composta da:

\begin{itemize}
\item Machiavellismo: Caratterizzato da manipolazione interpersonale, cinismo, distacco emotivo e orientamento strategico al raggiungimento dei propri obiettivi
\item Narcisismo: Caratterizzato da grandiosità, bisogno di ammirazione, senso di superiorità e forte focalizzazione su sé stessi.
\item Psicopatia subclinica: Caratterizzata da impulsività, ridotta empatia, scarsa ansia e tendenza a comportamenti antisociali non necessariamente clinici.
\end{itemize}

{\footnotesize (Paulhus \& Williams, 2002)}

\end{frame}


\section{Tipi di variabili}

\begin{frame}
	
	\begin{center}
	\Large  QUALITATIVE / CATEGORIALI
	\end{center}
\begin{itemize}
	\item sconnesse: $\{a,b,c\}$
	\item ordinate:$\{a,b,c\}$ con $a < b < c$
\end{itemize}

	\begin{center}
	\Large  QUANTITATIVE / METRICHE
\end{center}
\begin{itemize}
	\item Discrete: $\mathbb{Z}$
	\item Continue: $\mathbb{R}$
\end{itemize}
\end{frame}

\subsection*{Misura delle variabili}

\begin{frame}{Stevens (1946)}
	
	\small
	Si differenziano in base alla quantità di informazione che può essere ricavata
	
	\begin{itemize}
	\item Varibaili qualitative
	\begin{itemize}
	\item \textbf{Nominale}: Distingue un insieme di dati in diverse categorie
	\item \textbf{Ordinale}: Distingue un insieme di dati in diverse categorie che sono ordinabili a seconda della quantità di caratteristica posseduta 
	\end{itemize}
	
	\item Variabili quantitative
	
	\begin{itemize}
	\item \textbf{Intervalli equivalenti} Distingue un insieme di dati in diverse categorie ordinabili. La distanza tra le categorie è nota perché c'è un'unità di misura
	\item \textbf{A rapporti equivalenti}: Come la scala ad intervalli...ma lo 0 indica assenza di caratteristica (non è arbitrario ma è assoluto) 
	\end{itemize}
\end{itemize}
\end{frame}

\subsection*{Scala nominale}
\begin{frame}
\small
Comune di residenza, genere, patologia clinica ecc. Vale solo l'equivalenza ($=$ o $\neq$)

\begin{center}
	\includegraphics[width=.20\linewidth]{img/nominale.png}
\end{center}

Al posto di ``A'', ``B'', ``C'' si poteva scrivere $\alpha$, $\beta$ $\gamma$, 1,2,3 (ma i numeri valgono solo come etichette!)

Caratteristiche: 

\begin{itemize}
	\item  Categorie \textbf{distintive}: gli elementi che appartengono a categorie differenti vengono considerati di tipo diverso
	\item Categorie \textbf{mutualmente escludentesi}: ogni elemento può rientrare in una ed una sola categoria
	\item Categorie \textbf{Collettivamente esaustive}: tutti gli elementi vengono classificati nelle categorie della variabile, nessuno escluso
	
\end{itemize}
\end{frame}

\begin{frame}
	\begin{figure}
		\centering
		\includegraphics[width=0.4\linewidth]{img/regioni}
		\caption{Regione di nascita}
	\end{figure}
\end{frame}


\subsection*{Scala ordinale}
\begin{frame}

Titolo di studio, gradimento di un prodotto, valutazione di una malattia (e.g., alto, medio, basso) ecc. Vale sia l'equivalenza ($=$ o $\neq$) sia l'ordinamento ($<$ o $>$)
	
	\begin{center}
		\includegraphics[width=.20\linewidth]{img/ordinale.png}
	\end{center}
	
	Le etichette numeriche valgono per il loro ordine (non ha senso compiere operazioni tra di loro)
\end{frame}

\begin{frame}
	
	\begin{figure}
		\centering
		\includegraphics[width=0.5\linewidth]{img/ordinle}
		\caption{La staffetta 100$\times$4}
	\end{figure}
	
\end{frame}
\subsection*{Scala a intervalli equivalenti}
\begin{frame}
	
	Lo $0$ è arbitrario, ma c'è un'\emph{unita di misura} per cui le categorie si trovano alla stessa distanza. Temperatura in Celsius, Q.I., scale di atteggiamento ecc. Vale equivalenza ($=$ o $\neq$), ordinamento ($<$ o $>$) e differenza ($+$ o $-$): 
	\begin{center}
		\includegraphics[width=0.5\linewidth]{img/intervalli.png}
	\end{center}
	
	\begin{figure}
		\centering
		\includegraphics[width=0.6\linewidth]{img/termometroMulti.jpg}
	\end{figure}
	
\end{frame}

\begin{frame}
	\begin{columns}[T]
		\begin{column}{.50\linewidth}
		\begin{figure}
			\centering
			\includegraphics[width=\linewidth]{img/termometri}
		\end{figure}
		\end{column}
		
			\begin{column}{.50\linewidth}
Se ci sono 0°C, non vuol dire che non c'è calore! \\
Se ieri c'erano 5°C e oggi ce ne sono 10°C, possiamo dire che oggi fa più caldo di ieri e che ci sono 5°C più di ieri\\
Il fatto che lo 0 non sia arbitrario non ci permette di dire che oggi c'è il doppio del caldo di ieri! \\
Trasformando le due temperature in Fahrenheit si ottiene 41°F e 50°F... e la seconda non è più il doppio della prima!
		\end{column}
	\end{columns}

\end{frame}


\subsection{Scala a rapporti equivalenti}
\begin{frame}
	Lo 0 è assoluto, indica assenza del fenomeno. Peso, altezza, valori diagnostici. Vale equivalenza ($=$ o $\neq$),  ordinamento ($<$ o $>$), differenza ($+$ o $-$) e rapporto ($\times$ o $\div$)
	
	\begin{center}
		\includegraphics[width=0.40\linewidth]{img/rapporti.png}
	\end{center}
	
	
	% TODO: \usepackage{graphicx} required
	\begin{figure}
		\centering
		\includegraphics[width=0.4\linewidth]{img/bilancia}
	\end{figure}
	
\end{frame}

\begin{frame}
	% TODO: \usepackage{graphicx} required
	\begin{figure}
		\centering
		\includegraphics[width=0.5\linewidth]{img/scale1}
	\end{figure}
\end{frame}


\section[Frequenze]{Il dato elementare: Le Frequenze}

\subsection*{Frequenze}

\begin{frame}

		\begin{table}
				\centering
				\begin{tabular}{|c | c | c |}
				\hline
				modalità $x_i$ & $f_i$ & 	$f_i\% = (f_i/n) \cdot 100$ \\
				\hline
				$x_1$ & $f_1$ & $f_1\%$ \\
				$x_2$ & $f_2$ & $f_2\%$ \\
				$\vdots$ & $\vdots$ & $\vdots\%$ \\
				$x_k$ & $f_k$ & $f_k\%$ \\
				\hline
			   & $n$ & $100$ \\
			   \hline
				\end{tabular}
			\end{table}
			
\vspace{3mm}
Frequenza assoluta:	$f_i \ge 0$ interi e $\sum_{i=1}^{k} f_i = n$

Frequenza relativa: 	$f_i\% = \dfrac{f_i}{n} \cdot 100$ ($0 \leq f_i\% \leq 100$ e $\sum_{i=1}^{k} f_i\% = 100$)
\end{frame}

\subsection*{Frequenze Cumulate}
\begin{frame}
	\begin{itemize}
			\item \textbf{frequenza cumulata}: numero/frazione di
		unità statistiche che presentano una data caratteristica
		``minore o uguale'' alla corrente ($F_i$ o $F_i\%$): 
		
		\vspace{3mm}
		\begin{quote}
			$F_i$ frequenze \textbf{assolute cumulate} =
			
						\vspace{3mm}
			\centering
			
			$= f_1 + f_2 + \ldots + f_i = \sum_{j=1}^{i} = f_j$
			
			\vspace{3mm}
			$F_1 = f_1$ e $F_k = n$
		\end{quote}
		
		\begin{quote}
			$F_i\% = $ frequenze \textbf{relative cumulate} = 
			
						\vspace{3mm}
			\centering
			
			$= f_1\% + f_2\% + \ldots + f_i\% = \sum_{j=1}^{i} = f_j\%$
			
			\vspace{3mm}
			$F_1\% = f_1\%$ e $F_k\% = 100$
			
		\end{quote}
		
	\end{itemize}
\end{frame}

\begin{frame}
	La \textbf{frequenza cumulata} è una formula ricorsiva: 


\begin{block}{}
	
$F_i$ = frequenze \textbf{assoulte cumulate}: 

\begin{equation*}
	F_i = F_{i-1} + f_i
\end{equation*}


	
$F_i\%$ = frequenze \textbf{relative cumulate}: 

\begin{equation*}
	F_i\% = F_{i-1}\% + f_i\%
\end{equation*}
\end{block}




\begin{alertblock}
	
N.B: le frequenze cumulate si calcolano SOLO quando i caratteri presentano un ordinamento (e quindi non si calcolano per i caratteri sconnessi!!)
\end{alertblock}
\end{frame}

\begin{frame}
	\begin{exampleblock}{Opinione riguardo un'affermazione}
		\begin{spacing}\Factor
			
			\begin{table}
				\centering
				\begin{tabular}{|c | c | c| c | c|}
					\hline
					$x_i$ & $f_i$ & $F_i$&	$f_i\%$ & $F_i\%$ \\
					\hline
					Totalmente in disaccordo & $10$ & 10 & $24.4\%$ & $24.4$\%\\
					Abbastanza in disaccordo & $8$ &18 & $19.5\%$ & $24.4\% + 19.5\% = 43.9$\%\\
					Indifferente & $6$ &24& $ 14.6\%$ & $43.9\% + 14.6\% = 58.5$\% \\
				Abbastanza d'accordo & $14$&38 & $34.2\%$ & $58.5\% + 34.2\% = 92.7$\%\\
					Totalmente d'accordo & $3$ &41& $7.3\%$ & $92.7\% + 7.3\% = 100$\% \\

					\hline
					& $41$ & &$100$  \\
					\hline
				\end{tabular}
			\end{table}
		\end{spacing}
	\end{exampleblock}
\end{frame}

\section{Indici di posizione}


\begin{frame}
\begin{itemize}
	\item \textbf{indici sintetici} che evidenziano le caratteristiche essenziali della distribuzione della variabile
	
	\vspace{3mm}
	\begin{quote}
		
		Sono migliori gli studenti maschi o le femmine ?
		
		Hanno un’autostima più alta gli studenti di economia o di psicologia?
	\end{quote}
\vspace{3mm} 

\item Si confrontano categorie che rappresentano i livelli o valori tipici

\item Consentono di sintetizzare un insieme di misure tramite un unico valore ``rappresentativo'' $\rightarrow$ indice che riassume o descrive i dati e dipende dalla scala di misura dei dati in oggetto
\end{itemize}

\end{frame}

\subsection{Indici di posizione tipici}

\begin{frame}

\begin{itemize}
	\item moda
	\item mediana
	\item quartili/percentili
	\item media aritmetica
\end{itemize}

	Ogni variabile statistica ha l’indice di posizione adeguato: non tutti gli indici si possono calcolare per ogni carattere $\rightarrow$ \textbf{indice di posizione adeguato}
\end{frame}


\begin{frame}
	
	\begin{columns}
\begin{column}{.30\linewidth}

Variabile misurata su \textbf{scala nominale}

	\vspace{3mm}

Variabile misurata su \textbf{scala ordinale}
		\vspace{3mm}
Variabile quantitativa
	\end{column}

\begin{column}{.10\linewidth}
	$\rightarrow$
	
		\vspace{3mm}
	$\rightarrow$ 
	
		\vspace{3mm}
	$\rightarrow$  
\end{column}

\begin{column}{.30\linewidth}
	MODA
	
		\vspace{3mm}
	MEDIANA
	
		\vspace{3mm}
	MEDIA 	o MEDIANA
\end{column}


	\end{columns}

\end{frame}

\subsection*{Scala nominale}

\begin{frame}
	Categoria che presenta la massima frequenza
	
	\vspace{3mm}
	
	\begin{center}
		$Mo(X) = max(f_i)$ \\ (categoria con massimo valore di $f_i$)
	\end{center}
	
	
	\vspace{3mm}
	\begin{alertblock}
		
		N.B: Si può calcolare per le variabili misurate a qualsiasi livello, anche se di fatto viene calcolata solo per le variabile categoriali, in quanto per altri livelli di misurazione si
		possono calcolare altri indici più informativi
	\end{alertblock}
\end{frame}

\begin{frame}
\begin{columns}[T]
\begin{column}{.50\linewidth}
\begin{exampleblock}{Esempio 1}
		\begin{table}
			\centering
		
			\begin{tabular}{|c | c |}
				\hline
				 & $f_i$\\
				\hline
				\textbf{idoneo}  & $\mathbf{19}$ \\
				II scelta & 11\\
				difettoso & 4 \\
				scarto & 2\\
				\hline
				& 26 \\
				\hline
			\end{tabular}
		\end{table}

\vspace{3mm}
\centering
$max(f_i) =19$  $\Longrightarrow$ moda $=Mo(X)$=idoneo
	
	\end{exampleblock}
\end{column}

\begin{column}{.50\linewidth}
	\begin{exampleblock}{Esempio 2}
			
			\begin{table}
				\centering
				\begin{tabular}{|c | c |}
					\hline
					Stato civile & $f_i$\\
					\hline
						Celibe/Nubile & 141\\
					Coniugat*  & 200 \\
					\textbf{Divorziat*} & $\mathbf{249}$\\
					Vedov* & 115 \\
					\hline
					& 695 \\
					\hline
				\end{tabular}
			\end{table}

		\vspace{3mm}
		
		$max(f_i) =249$  $\Longrightarrow$ moda $=Mo(X)$=Divorziato
		
	\end{exampleblock}
\end{column}

\end{columns}
	
\end{frame}

\subsection*{Scala Ordinale}

\begin{frame}
Categoria/valore che occupa la posizione centrale o mediana ($Pos_{Me}$) nella distribuzione ordinata dei dati
	
	\begin{itemize}
		\item preceduta da almeno 50\% dei casi
		\item superata da almeno 50\% dei casi
	\end{itemize}

\begin{columns}
	
	\begin{column}{.50\linewidth}
		
		Numero dati dispari
		
		\includegraphics[width=\linewidth]{img/medianaDispari.png}
	\end{column}

	\begin{column}{.50\linewidth}
		Numero dati pari
		
	\includegraphics[width=\linewidth]{img/medianaPari.png}
\end{column}
\end{columns}

\begin{equation*}
		Pos_{Me} = \dfrac{(n+1)}{2}
	\end{equation*}


Come si individua sulla distribuzione di frequenza la posizione $\dfrac{(n+1)}{2}$?

Nella distribuzione delle \underline{ FREQUENZE CUMULATE}:  si individua la prima frequenza cumulata maggiore o uguale della posizione cercata
\end{frame}


\begin{frame}
	\begin{exampleblock}{ESEMPIO 1: livello di Istruzione di un
			campione di 203 soggetti}
		
	\begin{columns}
		\begin{column}{.50\linewidth}
			\begin{spacing}\Factor
				\begin{table}
					\centering
					\begin{tabular}{|c|c|c|}
						\hline
					Istruzione	& $f_i$ & $F_i$ \\
					\hline
						Elementari & 12 & 12 \\
						Scuole Medie & 17 & 29 \\
						Scuole superiori & 123 & 152 \\
						Laurea & 51 & 203 \\
						\hline
						Totale ($n$) & 203 & \\
						\hline
					\end{tabular}
				\end{table}
			\end{spacing}
		\end{column}
	
\begin{column}{.50\linewidth}
	$Pos_{Me} = \dfrac{(n+1)}{2} = \dfrac{203+1}{2} = Pos_{Me} = 102$
	
	\vspace{5mm}
	
	Si deve andare a cercare la $F_i$ subito superiore a $102$. In questo caso corrisponde alla Scuola superiore, la cui $F_i = 152$. 
\end{column}

	\end{columns}
		\vspace{5mm}
		\centering
		$Me(X) = X_{(n+1)/2} = X_{102} =$ \emph{Scuole Superiori}
	\end{exampleblock}
\end{frame}

\subsection*{Variabili quantititive: Percentili}
\begin{frame}{Percentili}
	Categorie/valori che dividono la
	distribuzione di frequenza
	\underline{ordinata} in più parti.
\begin{quote}
	Qual è il reddito familiare che divide il 25\% dei più poveri dal restante 75\% ?
	
	Qual è la soglia di reddito oltre cui sta la fascia dei più
	abbienti ?
	
	Quali sono le persone che hanno performato al meglio nell'ultimo semestre?
\end{quote}
	
\end{frame}

\begin{frame}
	\begin{columns}
		\begin{column}{.20\linewidth}
		\textbf{quartili}
	
		
		\textbf{decili}
		
		\textbf{percentili}
		\end{column}
	
	\begin{column}{.10\linewidth}
		$\rightarrow$
	
		$\rightarrow$
	
		$\rightarrow$
		
	\end{column}

\begin{column}{.50\linewidth}
	dividono in \textbf{4} parti la distribuzione
	
	
	dividono in \textbf{10} parti la distribuzione
	
	dividono in \textbf{100} parti la distribuzione 
\end{column}
	\end{columns}
\end{frame}

\begin{frame}
	Il percentile $x_p$ di ordine $p$ è
	quella categoria/valore che è:
	
	\begin{itemize}
	\item preceduta da almeno $p$\% dei casi
	\item superata da almeno $(1-p)$\% dei casi
	\end{itemize}

	\begin{columns}
	\begin{column}{.20\linewidth}
		\textbf{quartili}

		
		\textbf{decili}
		
		\textbf{percentili}
	\end{column}
	
	\begin{column}{.10\linewidth}
		$\rightarrow$
		
		$\rightarrow$
		
		$\rightarrow$
		
	\end{column}
	
	\begin{column}{.50\linewidth}
		percentili di ordine 25 - 50 - 75
		
		
	percentili di ordine 10 - 20 - $\ldots$ - 90
		
			percentili di ordine 1 - 2 - $\ldots$ - 98 - 99
	\end{column}
\end{columns}
\end{frame}

\begin{frame}
	Per il calcolo di quartili, decili e percentili il procedimento è
	analogo a quello della Mediana, che di fatto è un caso
	particolare di questi indici.
	
	\vspace{3mm}
	L’unica differenza riguarda il calcolo della \textbf{Posizione} ($Pos$)
	
	In generale la Posizione di ordine $p$ (in centesimi) si ottiene:
	
	\begin{equation*}
		Pos_p = \dfrac{n+1}{100}\cdot p
	\end{equation*}
\end{frame}

\begin{frame}

\begin{exampleblock}
		
		15 soggetti hanno espresso il loro grado di
		soddisfazione (punteggio da 1 a 7)
		relativamente all’erograzione di un servizio.
		I risultati sono:
		\begin{tabular}{c c c c c c c c c c c c c c c}
				1 & 5& 4& 6& 7& 2& 5 &6 &3 &1 & 2& 4& 4& 7& 7\\
		\end{tabular}

		Costruire la tabella delle frequenze (assolute e cumulate) e trovare il 1°, 2° e 3° quartile
	\end{exampleblock}

\end{frame}

\subsection*{Variabili Quantitative}

\begin{frame}

La media è una sorta di baricentro della distribuzione dei dati. Data una variabile $X$ composta da $n$ valori:  

\begin{equation*}
\bar{X} = \dfrac{\sum_{i = 1}^{n}x_i}{n}
\end{equation*}

\vspace{1.5mm}

\centering
\begin{tikzpicture}[scale=1.2]

\def\data{0.8,1.2,1.6,2.4,2.7,3.3,3.6,4.5,5.2}
\def\mean{2.81}

% Tavola
\draw[line width=2pt] (0.5,0.3) -- (6.5,0.3);

% Fulcro
\draw[fill=gray!40]
(\mean,0) -- ++(-0.35,-0.7) -- ++(0.7,0) -- cycle;
\node[below] at (\mean,-0.7) {Media};

% Puntini = osservazioni
\foreach \x in \data {
    \fill[blue!60] (\x,0.45) circle (0.12);
}

% Asse
\draw[->, thin, gray] (0,0) -- (7,0) node[right] {Valori};

\end{tikzpicture}

\flushleft 
\small 
I dati: $X = (0.8,1.2,1.6,2.4,2.7,3.3,3.6,4.5,5.2)$, la media $\bar{X} = 2.811$
\end{frame}



\begin{frame}
\centering
\scalebox{.80}{
\begin{tikzpicture}[scale=0.9]

\def\data{1.18,1.02,1.10,1.16,1.01,1.10,1.05,1.12,1.45,1.52,1.54,1.55,10.06,14.46}
\def\mean{2.809}
\def\angle{8}

% ---- calcolo minimo e massimo ----
\pgfmathsetmacro{\xmin}{1.01}
\pgfmathsetmacro{\xmax}{14.46}

% piccolo margine estetico
\pgfmathsetmacro{\leftend}{\xmin - 0.5}
\pgfmathsetmacro{\rightend}{\xmax + 0.5}

% Fulcro
\draw[fill=gray!40]
(\mean,0) -- ++(-0.4,-0.7) -- ++(0.8,0) -- cycle;
\node[below] at (\mean,-0.7) {Media};

% Tavola + dati ruotati attorno al fulcro
\begin{scope}[rotate around={-\angle:(\mean,0.3)}]

    % Tavola che copre TUTTI i dati
    \draw[line width=2pt] (\leftend,0.3) -- (\rightend,0.3);

    % Puntini
    \foreach \x in \data {
        \fill[blue!60] (\x,0.45) circle (0.12);
    }

\end{scope}

% Asse
\draw[->, thin, gray] (\leftend-0.5,0) -- (\rightend+0.5,0) node[right] {Valori};

\end{tikzpicture}
}

\flushleft 
\small 

I dati: $X = (1.18,1.02,1.10,1.16,1.01,1.10,1.05,1.12,1.45,1.52,1.54,1.55,10.06,14.46)$, la media $\bar{X} = 2.809$

\end{frame}

\end{document}